% Created 2016-12-25 Sun 18:35
\documentclass[11pt]{article}
\usepackage[utf8]{inputenc}
\usepackage[T1]{fontenc}
\usepackage{fixltx2e}
\usepackage{graphicx}
\usepackage{longtable}
\usepackage{float}
\usepackage{wrapfig}
\usepackage{soul}
\usepackage{textcomp}
\usepackage{marvosym}
\usepackage{wasysym}
\usepackage{latexsym}
\usepackage{amssymb}
\usepackage{hyperref}
\usepackage{graphicx}
\tolerance=1000
\providecommand{\alert}[1]{\textbf{#1}}

\title{IED Project Report}
\author{Aditya Chetan | Siddharth Yadav | Anant Sharma | Shwetank Shrey}
\date{20/12/2016}
\hypersetup{
  pdfkeywords={},
  pdfsubject={},
  pdfcreator={Emacs Org-mode version 7.9.3f}}

\begin{document}

\maketitle

\setcounter{tocdepth}{3}
\tableofcontents
\vspace*{1cm}

\newpage
\section{\textbf{Abstract}}
\label{sec-1}


Despite the advanced technology that we possess today, there are still places on this Earth, and the Space beyond that remain inaccessible to us because of the harsh conditions there. For example, according to the US Department of Commerce, we have explored just 5\% of the oceans on the Earth. This inability has been depriving us of precious knowledge and information of the world around us. However, we can overcome this hurdle if we were to replace humans with mechanized robots. This project is a step in this direction. The aim is to develop a pair of what we are calling at this stage as, ``Shadow-Bots''. These robots will emulate and interpret their operator's actions and gestures to give appropriate responses. We aim to make these robots participate in a mock bout of boxing, to see how accurately can the robot track its operators' actions and interact with other robots and its environment. The current idea is to make use of Kinect sensors to relay the driver's actions to the robot. With the help of this project, we seek to make robots the primary tool for humans, whether it be exploration, warfare or manufacturing.

\newpage
\section{\textbf{Introduction}}
\label{sec-2}


Up until March 24, 2016, 879 Indian soldiers had lost their lives^{\href{http://www.thehindu.com/news/national/in-siachen-869-army-men-died-battling-the-elements/article7978149.ece}{1}}, guarding the borders of their country, on the deserted front that is Siachen. Such a waste of human lives, although deemed necessary, still seems appalling. Similarly, it is almost funny how we, humans have set foot on the moon but are yet to completely explore our own planet Earth. For instance, we have explored just 5\% of the Earth’s oceans to-date^{\href{http://oceanservice.noaa.gov/facts/exploration.html}{2}}. However, this is not without good reason.


Most of these places provide conditions that cannot be withstood by normal human beings. The technology required to provide man a safe stay in these places does not come cheap, and still it is not completely safe. So how do we work around this hurdle? Replacing man with machines seems a good answer at first glance. But here is the problem: we are still a long way off from developing a self-sufficient autonomous robotic being. Also, in case we are looking to completely remove human presence from areas as critical as warfare or exploration, there would be little room for errors. So what is the next best-possible solution?


In this case, what seems to be a better option is developing systems that can allow robots to emulate human gestures based on feedback that they provide to the operator of these machines. If such a system were to be developed, than neither would we be completely removing human presence, nor would we be endangering human lives! Our project is a step in the direction of developing such systems.
\section{\textbf{Proposal}}
\label{sec-3}


The idea for the prototype is to construct two “Shadow-Bots”, or action-following robots and to have a bout of boxing between them. We plan to have the robots controlled by KInect sensors. 

The boxing match will help us analyse how effectively two robots being controlled by humans can interact with each other. Not only this, it will be a lot of fun as well!
\section{\textbf{How it will help?}}
\label{sec-4}


The project that we are going to do is pretty simple in its application compared to the way it can be used.Yet it does not, in any way, diminish its significance. Whatever results we obtain from this project will lay a solid foundation for future projects in this area. 

It is very sad that in our own solar system there is a planet next to us and it has a very slow moving Mars Rover that is responsible for its exploration. There are planets like Jupiter where it is even harder to have such a Rover in it. Why go that far? We have only explored 5\% of the Earth’s Oceans till date. But if we have humanoid robots who can shadow our movements, we can explore much more than we can have today. 

Well there is no limit to this, Humanoid bots shadowing our movements can be a lot of fun as well! If the prototype is implemented properly, in no time, Engineers work on earth and humanoid robots shadow them to create amazing buildings and architecture on the moon. We can probably see football matches on Mars and then maybe we can actually start looking for a planet which will save us when there are no natural resources left on earth to exploit. 

Before thinking about the people who may risk their lives, let us think about the people who lose their lives. Let us face it that humans are never going to live in peace and every country needs an army. But no country ever wants people from their own country to die for them. So it would be much better if the war was not between humans but machines shadowing humans.

Our prototype also has an interesting application which is even implementable. The world has a lot of fighting sports like boxing, karate, etc. If a karate fighter is practicing his moves, and we shadow his moves to a robot, he might be able to correct his mistakes when he sees himself make those mistakes in real-time. Also, after a lot of warnings about not trying the sports at home, people still do and get painful injuries. With our shadow bot fighters, they can do it absolutely free of danger.
\section{\textbf{Technical Aspects}}
\label{sec-5}

\begin{itemize}
\item The flow of data in the prototype will start with the Kinect sensors that will record the necessary information and then send it to the computer.
\item With the help of some middleware, to communicate with the Kinect, we will calculate the angles by which the arms of the prototype must be rotated.
\item We will wirelessly communicate the degree of rotation to the microcontroller responsible for controlling the robot.
\item The microcontroller will then relay the signal to the motors to rotate the limbs of the robot by the required amount.
\end{itemize}

For more details about the technical aspects, kindly have a look at Diagram \ref{fig:dig11}.

\begin{figure}
  \includegraphics[width=\linewidth]{./ied\_chart.png}
  \caption{Diagram 1}
  \label{fig:dig1}
\end{figure}
%\href{file://./home/aditya/Desktop/ied_chart.png }{/home/aditya/Desktop/ied\_chart.png }
\section{\textbf{Challenges}}
\label{sec-6}

Following is a list of challenges that we anticipate while working on this project:


\begin{itemize}
\item The most difficult challenge that we anticipate is coming up with a decently flexible and sturdy mechanical structure for the prototype model.
\item The motors that would be used to drive the robot’s  limbs would take some time to bring them to the desired configuration. Reducing this time lag would also be a challenge.
\item There would be about 10-12 motors in the prototype. Controlling and synchronizing them together would also be challenging.
\item We also plan to use Kinect sensors in order to drive these robots. So we will be learning how to use libraries associated with Kinect sensors.
\end{itemize}
\section{\textbf{Conclusion}}
\label{sec-7}


We hope to come up with interesting observations from this project that will be instrumental in helping develop future technologies that work on the same idea.

\end{document}
